\documentclass{beamer}
\usepackage{listings} 
\usepackage{color}
\usepackage[T1]{fontenc}


\usetheme{Frankfurt}
\usecolortheme{dove}

\title{Esame Telerilevamento Geo-Ecologico}
\institute{Alma Mater Studiorum - Università di Bologna\\Telerilevamento Geo-Ecologico}
\author{Studente: Leonardo Farinatti\\Docente: Duccio Rocchini }
\date{A.A. 2022/2023}
\logo{\includegraphics[height=1.5cm]{log.png}}


\begin{document}

\maketitle


\AtBeginSection[] 
{
\begin{frame}
\frametitle{Indice}
\tableofcontents[currentsection]
\end{frame}
}


\section{Scopo del progetto}

\begin{frame}{Scopo del progetto}
\begin{itemize}
    \item  Analizzare la perdita di manto nevoso subita dal ghiacciaio Mendenhall attraverso lo studio di due immagini risalenti al 1984 ed al 2023
\end{itemize}
\end{frame}

\section{Inquadramento}

\begin{frame}{Inquadramento geografico}
\begin{itemize}
    \item Il ghiacciaio Mendenhall è situato nell'omonima valle, a circa 19 km dal centro di Juneau, capitale dell'Alaska, nella zona sud-orientale dello stato statunitense. 
\end{itemize}
\includegraphics[width=1\textwidth]{Inquadramento_geografico.jpg}
    \centering
\end{frame}

\section{Metodi di analisi}

\begin{frame}{Metodi}
\begin{itemize}
    \item Caricamento e preparazione iniziale delle immagini
    
    \item \pause Calcolo pca (principal component analysis) delle due immagini
    
    \item \pause Calcolo della perdita di manto nevoso
    
    \item \pause Calcolo della deviazione standard per mostrare i punti in cui lo scioglimento del ghiacciaio è stato maggiore
\end{itemize}    
\end{frame}

\section{Analisi immagini}

\begin{frame}{Script usato}
\begin{itemize}
    \item Per prima cosa è stato necessario \textbf{importare} le 2 immagini in R e \textbf{assegnarle} un nome
\end{itemize}
\bigskip
\small mendenhall\_1984 <- brick("mendenhallglacier\_tm5\_1984230\_lrg.jpg")\\\small mendenhall\_2023 <- brick("mendenhallglacier\_oli\_2023209\_lrg.jpg")
\end{frame}

\begin{frame}{Componenti Mendenhall 1984}
    \includegraphics[width=7.5cm, height=7.5cm]{mendenhall_1984_components.jpg}
    \centering
\end{frame}

\begin{frame}{Componenti Mendenhall 2023}
    \includegraphics[width=7.5cm, height=7.5cm]{mendenhall_2023_components.jpg}
    \centering
\end{frame}

\begin{frame}{Script usato}
\begin{itemize}
    \item Successivamente è stata utilizzata la funzione \textbf{"plotRGB"} per poter visualizzare le immagini con i colori reali delle immagini satellitari di partenza
\end{itemize}
\bigskip
\small plotRGB(mendenhall\_1984,r=1,g=2,b=3, stretch="lin")\\
plotRGB(mendenhall\_2023,r=1,g=2,b=3, stretch="lin")
\end{frame}

 \begin{frame}{Script usato}
\begin{itemize}
    \item Attraverso la funzione \textbf{"par"} si è creato lo spazio grafico necessario per affiancare le due immagini per poterle \textbf{confrontare}
\end{itemize}
\bigskip
\small par(mfrow=c(1,2))\\
plotRGB(mendenhall\_1984,r=1,g=2,b=3, stretch="lin")\\
plotRGB(mendenhall\_2023,r=1,g=2,b=3, stretch="lin")
\end{frame}

\begin{frame}{Immagini RGB comparate}
    \includegraphics[width=10cm, height=4.5cm]{mendenhall_1984 VS mendenhall_2023.jpeg}
    \centering
\end{frame}

\section{PCA}

 \begin{frame}{Script usato}
\begin{itemize}
    \item Il seguente codice è stato utilizzato per il calcolo della \textbf{PCA} (principal component analysis) di entrambe le immagini
\end{itemize}
\bigskip
\small sample\_1984 <- sampleRandom(mendenhall\_1984, 10000)\\pca\_1984 <- prcomp(sample\_1984)\\summary(pca\_1984)\\
\bigskip
\small sample\_2023 <- sampleRandom(mendenhall\_2023, 10000)\\pca\_2023 <- prcomp(sample\_2023)\\summary(pca\_2023)
\end{frame}

\begin{frame}{Componenti PCA 1984}
    \includegraphics[width=7cm, height=7cm]{pca_1984.jpeg}
    \centering
\end{frame}

\begin{frame}{Componenti PCA 2023}
    \includegraphics[width=7cm, height=7cm]{pca_2023.jpeg}
    \centering
\end{frame}

\begin{frame}{Script usato}
\begin{itemize}
    \item Dopo aver calcolato la PCA si è proceduto alla realizzazione dei due plot con le \textbf{"principal component map"} del 1984 e del 2023
\end{itemize}
\bigskip
\small pci\_1984 <- predict(mendenhall\_1984, pca\_1984, index=c(1:2))\\pci\_2023 <- predict(mendenhall\_2023, pca\_2023, index=c(1:2))\\
\bigskip
pcid\_1984 <- as.data.frame(pci\_1984[[1]], xy=T)\\ggplot() +
geom\_raster(pcid\_1984, mapping = aes(x=x, y=y, fill=PC1)) +
scale\_fill\_viridis() +
ggtitle("pci 1984")\\
\bigskip
pcid\_2023 <- as.data.frame(pci\_2023[[1]], xy=T)\\
ggplot() +
geom\_raster(pcid\_2023, mapping = aes(x=x, y=y, fill=PC1)) +
scale\_fill\_viridis() +
ggtitle("pci 2023")
\end{frame}

\begin{frame}{Principal component map 1984}
    \includegraphics[width=9cm, height=7cm]{pci_1984.jpeg}
    \centering
\end{frame}

\begin{frame}{Principal component map 2023}
    \includegraphics[width=9cm, height=7cm]{pci_2023.jpeg}
    \centering
\end{frame}

\section{Perdita manto nevoso}

\begin{frame}{Script usato}
\begin{itemize}
    \item Per stimare la perdita di manto nevoso è stata calcolata la \textbf{differenza} tra la prima componente delle immagini del 1984 e del 2023
\end{itemize}
\bigskip
\small b1\_1984 <- mendenhall\_1984$mendenhallglacier\_tm5\_1984230\_lrg\_1\\b1\_2023 <- mendenhall\_2023$mendenhallglacier\_oli\_2023209\_lrg\_1\\
\bigskip
diff <- b1\_1984 - b1\_2023
\end{frame}

\begin{frame}{Script usato}
\begin{itemize}
    \item E' stata poi creata una nuova \textbf{color palette} per visualizzare meglio il plot della differenza
\end{itemize}
\bigskip
\small cl <- colorRampPalette(c('yellow','black','red'))(100)\\
\bigskip
plot(diff, col=cl, main = "loss of snow")
\end{frame}

\begin{frame}{Differenza tra le due immagini}
    \includegraphics[width=8cm,height=8cm]{diff.jpg}
    \centering
\end{frame}

\section{Deviazione Standard}

\begin{frame}{Script usato}
\begin{itemize}
    \item Calcolo della deviazione standard attraverso la funzione \textbf{"focal"} per enfatizzare le aree che sono state colpite da un maggiore scioglimento del ghiacciaio
\end{itemize}
\bigskip
\small devst <- focal(diff, w=matrix(1/25, nrow=5, ncol=5), fun=sd)\\
\bigskip
cl2 <- colorRampPalette(c("blue","green","yellow","magenta"))(100)\\
plot(devst, col=cld, main="standard deviation")
\end{frame}

\begin{frame}{Mappa analisi deviazione standard}
    \includegraphics[width=8cm, height=8cm]{standard deviation map.jpg}
    \centering
\end{frame}

\section{Conclusioni}

\begin{frame}{Conclusioni}
\begin{itemize}
    \item Le immagini prodotte mostrano come, nell'intervallo di tempo dal 1984 al 2023, le dimensioni del ghiacciaio Mendenhall siano diminuite. In particolar modo lo scioglimento ha riguardato l'\textbf{area basale} del ghiacciaio e le sue zone più periferiche, mentre la porzione centrale è rimasta pressochè invariata
\end{itemize}
\end{frame}

\begin{frame}
\centering
\huge Grazie dell'attenzione
\end{frame}

\end{document}
